\documentclass[6008notes.tex]{subfiles}
\begin{document}
\graphicspath{ {images/conditioning/} }

\section{Conditioning on Events}

\subsection{Conditioning on Events Intro}

The next important concept is called conditioning. In practice, we often encounter this kind of situation in that we have a probability model for some situation, but then we made some observation or something happened, and it changed the probability model. For example, if you observe that it was raining today, then the probability of tomorrow being a rainy day would certainly change. If we observed some information about a certain company today, then the probability that its stock price would increase tomorrow would be different. How do we address this in our formal mathematical language? This is about conditioning. 

Now let's go back to this familiar picture of a general sample space, $\Omega$. Suppose we have a probability model all built up, $(\Omega, \mathbb {P})$. There are two events in general $A$ and $B$. They could have some intersection $A \cap B$. Depending on where the outcome falls, we say event $A$ occurs, event $B$ occurs, or both $A$ and $B$ occur.

Suppose that we have an observation that $A$ occurred. That means the outcome of the random experiment is $\omega \in A$. Now we should not be worrying about anything outside of $A$, because they don't happen anymore. The set $A$ becomes our new sample space: $\Omega \longrightarrow A$. We still don't know exactly where the outcome would be. And therefore, there should be a new sample space, a new model, and a new probability assignment $\mathbb{P}$ conditioned on $A$: $\mathbb{P} \longrightarrow \mathbb{P}(\cdot | A)$. This is the conditional probability, conditioned on ``event $A$ occurs''. 

For any possible outcome $\omega$ from the original picture, what is $\mathbb{P}(\omega | A)$? There could be two different cases:

\begin{itemize}
\item If $\omega \notin A$, $\mathbb{P}(\omega | A) = 0$. That's never going to occur anymore.

\item If $\omega \in A$, $\mathbb{P}(\omega | A) = \frac{\mathbb{P}(\omega)}{\mathbb{P}(A)}$.
\end{itemize}

Scaling with $\frac{1}{\mathbb{P}(A)}$ normalizes the conditional distribution so that $\sum_{\omega \in A} \mathbb{P}(\omega | A) = 1$.

\subsection{The Product Rule for Events}

We should start to avoid writing probability in this way. We should always think of probability of some kind of event. So when I write an $\omega$, I'm thinking of a single element event --- a set with only one element $\omega$. A better way of writing this is to look at the probability of another event $B$ given $A$ occurs. The only way for B to occur is that the random experiment gives an outcome in $A \cap B$. The conditional probability after normalizing is $\mathbb{P}(B | A) = \frac{\mathbb{P}(A \cap B)}{\mathbb{P}(A)}$. 

Alternatively, this can be written $\mathbb{P}(A \cap B) = \mathbb{P}(A) \cdot \mathbb{P}(B | A)$. This is called the ``product rule''


\subsection{Bayes' Theorem for Events}

\paragraph{Important note about dividing by probabilities:} We will often divide by probabilities. In videos, we might not always say this, but this is required: we cannot divide by 0. To ensure this, we will not condition on events that have probability 0.

Given two events $\mathcal{A}$ and $\mathcal{B}$ (both of which have positive probability), Bayes' theorem, also called Bayes' rule or Bayes' law, gives a way to compute $\mathbb {P}(\mathcal{A} | \mathcal{B})$ in terms of $\mathbb {P}(\mathcal{B} | \mathcal{A})$. This result turns out to be extremely useful for inference because often times we want to compute one of these, and the other is known or otherwise straightforward to compute.

Bayes' theorem is given by

{\centering$\mathbb {P}(\mathcal{A} | \mathcal{B}) = \frac{\mathbb {P}(\mathcal{B} | \mathcal{A}) \mathbb {P}(\mathcal{A})}{\mathbb {P}(\mathcal{B})}.$ \par}
 
The proof of why this is the case is a one liner:

{\centering$\mathbb {P}(\mathcal{A} | \mathcal{B}) \overset {(a)}{=} \frac{\mathbb {P}(\mathcal{A} \cap \mathcal{B})}{\mathbb {P}(\mathcal{B})} \overset {(b)}{=} \frac{\mathbb {P}(\mathcal{B} | \mathcal{A}) \mathbb {P}(\mathcal{A})}{\mathbb {P}(\mathcal{B})},$ \par}
 
where step $(a)$ is by the definition of conditional probability for events, and step $(b)$ is due to the product rule for events (which follows from rearranging the definition of conditional probability for $\mathbb {P}(\mathcal{B} | \mathcal{A})$).

\subsection{Practice Problem: Bayes' Theorem and Total Probability}

Your problem set is due in 15 minutes! It's in one of your drawers, but they are messy, and you're not sure which one it's in.

The probability that the problem set is in drawer $k$ is $d_k$. If drawer $k$ has the problem set and you search there, you have probability $p_k$ of finding it. There are a total of $m$ drawers.

Suppose you search drawer $i$ and do not find the problem set.

\begin{itemize}
\item (a) Find the probability that the paper is in drawer $j$, where $j \neq i$.

\item (b) Find the probability that the paper is in drawer $i$.
\end{itemize}

\paragraph{Solution:} Let $A_k$ be the event that the problem set is in drawer $k$, and $B_k$ be the event that you find the problem set in drawer $k$.

(a) We'll express the desired probability as $\mathbb {P}(A_ j|B_ i^ c)$. Since this quantity is difficult to reason about directly, we'll use Bayes' rule:

{\centering$\mathbb {P}(A_ j|B_ i^ c) = \frac{\mathbb {P}(B_ i^ c | A_ j) \mathbb {P}(A_ j)}{\mathbb {P}(B_ i^ c)}$ \par}
 
The first probability, $\mathbb {P}(B_ i^ c | A_ j)$, expresses the probability of not finding the problem set in drawer $i$ given that it's in a different drawer $j$. Since it's impossible to find the paper in a drawer it isn't in, this is just 1.

The second quantity, $\mathbb {P}(A_ j)$, is given to us in the problem statement as $d_j$.

The third probability, $\mathbb {P}(B_ i^ c) = 1-\mathbb {P}(B_ i)$, is difficult to reason about directly. But, if we knew whether or not the paper was in the drawer, it would become easier. So, we'll use total probability:

\begin{eqnarray*}
            \mathbb{P}(B_i) &=& \mathbb{P}(B_i|A_i)\mathbb{P}(A_i) + \mathbb{P}(B_i|A_i^c)\mathbb{P}(A_i^c) \\
            &=& p_i d_i + 0 (1-d_i)
\end{eqnarray*}

Putting these terms together, we find that

{\centering$\mathbb {P}(A_ j|B_ i^ c) = \frac{d_ j}{1-p_ i d_ i}$ \par}
 
\paragraph{Alternate method to compute the denominator $\mathbb {P}(B_ i^ c)$:} We could use the law of total probability to decompose $\mathbb {P}(B_ i^ c)$ depending on which drawer the homework is actually in. We have

\begin{eqnarray*}
        \mathbb{P}(B_i^c) &=& \sum_{k=1}^m
                       \underbrace{\mathbb{P}(A_k)}_{d_k}
                       \underbrace{\mathbb{P}(B_i^c|A_k)}_{\substack{1\text{ if }k\ne i,\\
                                                             (1-p_i)\text{ if }k=i}} \\
                  &=& \sum_{\substack{k=1,\\
                                     k\ne i}}^m d_k
                     + (1-p_i)d_i \\
                  &=& \sum_{k=1}^m d_k - p_i d_i \\
                  &=& 1 - p_i d_i.
\end{eqnarray*}

(b) Similarly, we'll use Bayes' rule:

{\centering$\mathbb {P}(A_ i | B_ i^ c) = \frac{\mathbb {P}(B_ i^ c | A_ i) \mathbb {P}(A_ i)}{\mathbb {P}(B_ i^ c)} = \frac{(1-p_ i) d_ i}{1 - p_ i d_ i}$ \par}
 
\paragraph{Take-Away Lessons:}

\begin{itemize}
\item Defining the sample-space is not always going to help solve the problem. (It's difficult to precisely define the sample space for this particular problem)

\item When in doubt of being able to precisely define the sample space, try to define events intelligently, i.e., in a way that you use what you're given in the problem.

\item The probability law of a probability model is a function on events, or subsets of the sample space, i.e., one can work with the probability law without knowing precisely what the sample-space (as a set) is.
\end{itemize}


\end{document}